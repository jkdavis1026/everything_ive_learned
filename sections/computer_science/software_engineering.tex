\chapter{Software Engineering}

I took this class in the Fall of 2025 with a professor named Dr. Lehr. Several students have described his personality as “Jekyll and Hyde,” and while I wholeheartedly agree, I’d add that his grading fits the same description.

That said, this course may be one of the most practical and applicable in my entire Computer Science degree. Why? Because it’s designed to simulate a real working environment. The class was divided into groups of three to five members, and each team was tasked with creating a project. My group decided to build a
\textbf{Music Recommendation App.}

The project was written in React and Python — React for the frontend, and Python for the backend. I spent most of my time working on the backend.

If you’re reading this and thinking, “What the heck is a frontend or backend?” here’s a quick explanation: the frontend is what you see; the backend is what makes what you see work. Think of Facebook — when you scroll through your newsfeed and see everyone posting cute cat photos or ranting about the latest political crisis, the frontend is what’s displayed on the screen. The backend is where all that content lives, stored in a massive database. Every post, comment, and like is just a piece of data among billions of others.

If you’re curious about how that data gets from Facebook’s servers to your screen in the blink of an eye, check out my notes on \textbf{Data Structures and Algorithms.}


\section{Music Recommendation App}
\subsection{The Idea Behind It}

